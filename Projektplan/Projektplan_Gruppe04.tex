\documentclass[a4paper,10pt]{article}

\usepackage[utf8]{inputenc}
\usepackage{ngerman}


\begin{document}

\title{\textbf{MMIS - Konstruktionsübung  Projektplan}}
\author{ Neubauer Georg, \\
Reiter Andreas, \\
Tschinkel Gerwald
}
\date{WS2009, \\
\today{}}
\maketitle

%\newpage

\section{Teilnehmer}
\begin{tabular}{|ccc|}
\hline
Name & Matrikelnummer & e-Mail \\
\hline
Neubauer Georg & 0530228 & georg.neubauer@student.tugraz.at \\
Reiter Andreas & 0530152 & andreas.reiter@student.tugraz.at \\
Tschinkel Gerwald & 0612771 & gerwald.tschinkel@student.tugraz.at \\
\hline
\end{tabular}

\section{Thema}
\begin{center}
\Large{\textbf{Android GPS Tracking Applikation}}
\end{center}

\section{Beschreibung}
Es wird eine Android Applikation erstellt, welche dem Benutzer ermöglicht seine
aktuelle Position (ermittelt über GPS / Mobilfunk Peilung) zu ermitteln und diese
via Webservice an ein schon bestehendes Internet Portal (www.crossingways.com) zu
senden, um es dort mit Freunden zu teilen. \\

\underline{In der Android Oberfläche gibt es folgende Möglichkeiten:}
\begin{itemize}
\item Beim erstmaligen Start wird nach Benutzernme/Passwort gefragt
\item Start / Stop des Life Trackings
\item Anzeige diverser Daten wie: GPS Position, Geschwindigkeit, Stoppuhr, ...
\end{itemize}

\underline{optional:}
\begin{itemize}
\item Einmaliges Senden der aktuellen Position.
\item Senden der Position via SMS (für Auslandsaufenthalte)
\item Abspeichern einer Reihe von Positionen als Track - z.B.: Radfahrtrip, autotrip, Wandern, Reise, ...
\item Photo Button (Macht direkt aus der Anwendung heraus Fotos - somit ist eine Geocodierung möglich)
\item Bearbeiten (Name, etc...) und Upload eines Tracks inklussive aller  wärend der Zeit gemachten Fotos auf das Portal www.crossingways.com
\end{itemize}

\section{Grobes Design}
Aufbauend auf einem bereits verfügbaren Tracking Clients für Windows Mobile wird ein ähnlicher client für
Google Android implementiert. Dazu wird ein UserInterface am Android benötigt. Die Serverseitige Implementierung
besteht bereits. Tracking Daten werden von der Applikation erstellt. Dies geschieht durch Positionsbestimmung über
das vorhandene GPS-Modul, bzw. über Mobilfunk Peilung. Die aktuellen Positionsdaten werden von der Applikation laufend
bestimmt und mitgespeichert (im Tracking Modus). Die aktuell gültige Position wird dem Benutzer am graphischen Interface
angezeigt. Der Tracking Modus ist optional und kann jederzeit gestartet und beendet werden. Aufgezeichnete Daten können auch
jederzeit angezeigt werden. Für die aktuelle Position können Photos hinterlegt werden, diese werden in die Datenstruktur
eingebunden. Zusätzlich wird noch eine Stopuhr Funktion und eine Geschwindigkeitsmessung implementiert. Aufgezeichnete
Daten können jederzeit an den TrackingServer (www.crossingways.com) gesendet werden. Dies geschieht entweder über eine
bestehende mobile Internetverbindung, oder optional per SMS-Versand.\\
Die Applikation setzt sich aus folgenden Teilen (Modulen) zusammen:
\begin{itemize}
 \item UserInterface: Anzeigen aktueller Daten, Tracking, Steuerung, Aufnahme von Photos, Initialiseren des Übermittelns
an den Server
 \item Interface zu GPS-Modul: Zuständig für die Bestimmung der akutellen Position
 \item lll
\end{itemize}



\end{document}

